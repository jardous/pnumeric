\documentclass{article}

\title{pNumeric module for Python}
\author{Jiri Popek}
\authoraddress{
        Email: \email{jiri.popek@gmail.com}
}


\begin{document}

\maketitle

\section{Introduction \label{intro}}
    pNumeric module for Python is set of functions and datatypes for signal
    processing focused on Kalman filter utilization.

    The Kalman filter is created from scratch and all support routines are
    also provided. So you do not need any additional libraries - Numeric, 
    etc. There is also FFT algorithm available.
  
    Because it deals with matrix algebra, some support algorithms had to be implemented. 
    There are functions for matrix manipulation and matrix operations - matrix
    multiplication, determinant computing, matrix inversion.
    
    No 3rd party libraries are needed while using pNumeric.
    
    Early versions has been tested even for PyS60 (Python for Symbian S60). There are build scripts
    still available.


\section{Installation \label{installation}}
  pNumeric uses distutils, so the installation on PC is pretty simple - just
  run:
  
  \begin{verbatim}
  python setup.py install
  \end{verbatim}
  
  Note that you'll need root privilegies to install it.
  You can just build it by typing
  
  \begin{verbatim}
  python setup.py install
  \end{verbatim}
  
  and copy \file{pnumeric.so} to the application's path.
  
  There are also build scripts for PyS60.

  Note that POSIX standard operating system is essential with build environment set.
  Preferred is Linux operating system with GNU GCC.


\section{Module level functions \label{modulefuncts}}
    
    Module pNumeric has several methods. The table gives a brief desctiption:
    
    \begin{tabular}{|l|l|} \hline
    \textbf{function name} & \textbf{Description} \\ \hline
    ones(n)      & returns Matrix of size n with all items set to 1 \\
    zeros(n)     & returns Matrix of size n with all items set to 0 \\
    eye(n)       & returns eye Matrix of size n with zeros everywhere except ones at main diagonal \\
    kf_process(...)  & Kalman filter function \\
    fft(Vector v)       & Fast Fourier Transform of v \\
    mean(Vector v)       & mean value of v \\
    rms(Vector v)       & Root Mean Square of v \\
    \end{tabular}


\section{Defined datatypes \label{datatypes}}
  
  \subsection{Vector \label{Vector}}
    Vector datatype is used to store measurements or other vectors. To create 
    Vector object just simply type:
    \begin{verbatim}
    from pnumeric import *
    >>> v = Vector([1, 2, 3])
    >>> v
    [1.0, 2.0, 3.0]
    \end{verbatim}
    
    Vector constructor accepts list or tuple with all numeric values.

    It is possible to get Vertor item by accessing it as list:
    
    \begin{verbatim}
    >>> v[2]
    3.0
    \end{verbatim}
    
    Also changing any of items is possible:
    \begin{verbatim}
    >>> v[1] = 88
    >>> v
    [1.0, 88.0, 3.0]
    \end{verbatim}
  
  \subsection{Matrix \label{Matrix}}
    Matrix is used to store matrixes and implements matrix algebra related
    functions.
    
    Use following syntax to create Matrix object
    
    \begin{verbatim}
    from pnumeric import *
    m2 = Matrix([[.1, .2], [-.2, .1]])
    m3 = Matrix([[.2, .4, .2], [-.2, .2, .0], [.2, .2, -.2]])
    \end{verbatim}
    Vector constructor accepts list or tuple with all numeric values.
    
    We can use also tuple instead of lists - be sure you specify the same 
    length of sublists (subtuples).

    Matrix object has following methods:
    
    \begin{tabular}{|l|l|} \hline
    \textbf{property name} & \textbf{Description} \\ \hline
    det      & returns determinant of the matrix \\
    inv      & returns inversion matrix \\
    \end{tabular}
    
    Matrix object has following properties:
    
    \begin{tabular}{|l|l|} \hline
    \textbf{property name} & \textbf{Description} \\ \hline
    rows      & returns number of Matrix columns \\
    cols      & returns number of Matrix columns \\
    shape     & returns an Matrix shape tuple (rows, cols) \\
    \end{tabular}

    It is possible to get Matrix item by accessing as list. This
    return matrix as a Vector type:
    
    \begin{verbatim}
    >>> m3[2]
    [-0.2, 0.2, 0.0]
    \end{verbatim}
    
    Also change any of items is possible:
    
    \begin{verbatim}
    >>> m3[1][1] = 66
    >>> m3
        0.2      0.4      0.2
        -0.2       66        0
        0.2      0.2     -0.2
    \end{verbatim}







\section{Kalman filter \label{kalmanfilter}}
    Kalman filter in pNumeric package is designed as general as possible. We can 
    filter MIMO systems with a nuber of state variables. Also updating of
    model after every time step is possible (EKF).
    
    Following example is to show Kalman filter usage. This is the simplest case of
    estimating a constant of 501.45 disturbed by white noise.
    \begin{verbatim}
    from pnumeric import *
    from random import gauss
    
    length = 30 # data length
    value  = 501.45 # constant value to be estimated
    t = range(length)
    # generate some noise around the value
    yv = [gauss(value, 9) for x in range(length)]
    
    y = Matrix([yv])
    u = Matrix([[0]*length]) # input values are not necessary - so create Matrix of zeros
    
    # define system matrixes
    A = Matrix([[1]])
    B = Matrix([[0]])
    C = Matrix([[1]])
    D = Matrix([[0]])
    
    # process and noise variations
    Q = Matrix([[0.5]])
    R = Matrix([[1.0]])
    
    # define initial values
    x0 = Matrix([[500.0]])
    P0 = Matrix([[9.0]])
    # process input by Kalman filter
    x_est, y_est, P_est = kf_process(A, B, C, D, y, u, x0, P0, Q, R)
    \end{verbatim}
    
    x_est now contains list of estimated states, y_est is list of estimated output values and P_est
    is list of covariance estimations.
    
    To update the model, define a Python function
    \begin{verbatim}
    # callback for updating the matrixes
    def callback(step, A, B, C, D, x):
        A[0][1] = step*dt
        B[0][0] = 0.5*(step*dt)**2
        B[1][0] = step*dt
        # reset the x
        x[0][0] = s0
        x[1][0] = v0
    \end{verbatim}
    
    This function parameters are step number an model parameters.
    Then we can simply put the function callback name as the last parameter to kf_process
    and filter will automaticaly call the function after every step.






\section{Fast Fourier Tranform \label{fft}}
    fft() function computes a spectrum of time series given by a Vector parameter.
    
    Example:
    \begin{verbatim}
    v = Vector([0, 1, 0, -1, 0, 1, 0, -1]) # create time series Vector
    out = fft(v)
    \end{verbatim}
    
    fft returns Vector containing sperctrum of time series.
    
\section{Statistical methods \label{statmeth}}
    pNumeric module contains some statistical functions:

    \begin{tabular}{|l|l|} \hline
    \textbf{function name} & \textbf{Description} \\ \hline
    mean(Vector v)     & returns mean value of v \\
    rms(Vector v)      & returns Root Mean Square of v \\
    \end{tabular}
\end{document}
